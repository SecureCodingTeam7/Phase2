\chapter{Persistent XSS}

\section{Observation}

We discovered that the field "Name" in sc-course/views/users/register.php is vulnerable to persistent Cross-Site-Scripting(XSS) with the following steps:
\begin{enumerate}
 \item Go to sc-course/views/users/register.php
 \item Insert malicious js code in textfield
\end{enumerate}
 If the Admin goes to sc-course/views/approve\_users.php the user information about new registrations are  loaded out of the database. In our case the malicious code is executed when displaying the name.

There are many possibilities to exploit this vulnerability. For example the admin could be redirected to a fake site where all his inputs will be stored.\newline


Likelihood: high \newline

Impact: medium-high\newline

Risk: high\newline

\section{Discovery}

This vulnerability was discovered by simply testing the possible inputs for all text fields. By inserting a simple alert() command, we checked if our input was sanitized before inserted into the database. Obviously that's not the case for the name field.

\section{Likelihood}
Exploiting this vulnerability requires basic technical skills. The attacker has to know about the basics of a XSS attack and basic java-script commands. It is possible to exploit this vulnerability from Internet as normal user.

\section{Implication}
A successful attack could lead to compromise of sensitive data. Also the attacker could log in as admin by accessing his session key.

\section{Recommendations}
To prevent Cross-Site-Scripting attacks, the user input has to be sanitized before inserted into the database. Go to \url{https://www.owasp.org/index.php/Top_10_2013-A3-Cross-Site_Scripting_(XSS)} for more information. \newline
This observation relates to the OWASP Top-10 issues A3 - Cross-Site-Scripting

