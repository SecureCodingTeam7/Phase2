\documentclass{article}
\usepackage{hyperref}

\begin{document}

\part*{Persistent XSS}

\section*{Observation}

We discovered that the field "Name" in sc-course/views/users/register.php is vulnerable to persistent Cross-Site-Scripting(XSS).
An attacker can enter any kind of java-script code in this text field. If the Admin goes to sc-course/views/approve\_users.php the user information about new registrations are  loaded out of the database. In our case the malicious code is executed when displaying the name.

There are many possibilities to exploit this vulnerability. For example the admin could be redirected to a fake site where all his inputs will be stored.\newline


Likelihood:  \newline

Impact: \newline

Risk: \newline

\section*{Discovery}

This vulnerability was discovered by simply testing the possible inputs for all text fields. By inserting a simple alert() command, we checked if our input was sanitized before inserted into the database. Obviously that's not the case for the name field.

\section*{Likelihood}
Exploiting this vulnerability requires basic technical skills. The attacker has to know about the basics of a XSS attack and basic java-script commands. It is possible to exploit this vulnerability from Internet as normal user.

\section*{Implication}
A successful attack could lead to compromise of sensitive data. Also the attacker could log in as admin by accessing his session key.

\section*{Recommendations}
To prevent Cross-Site-Scripting attacks, the user input has to be sanitized before inserted into the database. Go to \url{https://www.owasp.org/index.php/Top_10_2013-A3-Cross-Site_Scripting_(XSS)} for more information. \newline
This observation relates to the OWASP Top-10 issues A3 - Cross-Site-Scripting


\part*{MYSQL Injection}

\section*{Observation}

We discovered that the request \textit{host-ip/sc-course/views/transactions/tan\_entry.php?transaction\_id=x} is vulnerable to MYSQL Injections.
The attacker has to be logged in as user in order to create a new transaction . After inserting an account number and an amount for the new transaction the above-named request is sent. The user now would have to insert the TAN. At this point the parameter "transaction\_id can be used for all kind mysql  injections.


Likelihood: high\newline

Impact:      	high\newline

Risk: high \newline

\section*{Discovery}
This vulnerability was discovered by the tool sqlmap. 
\begin{enumerate}
\item \code{sqlmap -u "host-ip/sc-course/views/transactions/tan\_entry.php?transaction\_id=x"}
\end{enumerate}



\section*{Likelihood}
Assuming that the attacker has basic knowledge about MySql-Injection and knows how to use sqlmap, it's very likely that this vulnerability will be exploited.


\section*{Implication}
The vulnerability in combination with sqlmap leads to full compromise of all data, stored in the database. The attacker can hijack information about the names of databases, tables, columns of tables and can even dump whole tables.  In our special case, it was even possible to get the users' passwords in plaintext, because they weren't hashed.
You can also get the name and the password of the root user of the database. This was benifited by a weak password, whereby the plain password could be recovered from the received hash within seconds. This results in full write access to the database for the attacker.


\section*{Recommendations}
It's highly recommended, that all inputs for SQL queries are sanitized by escaping the ' character.
Another option is to create query objects via msqli instead of using simple query strings.
This objects will automatically sanitized all inputs. For more information, please go to \url{https://www.owasp.org/index.php/SQL\_Injection}

This observation is related to OWASP Top 10-A1-Injections




\end{document}