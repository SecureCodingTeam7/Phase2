\chapter{rsync Deamon}
\label{appendix:rsync}

In order to be able to exploit the Shell Injection vulnerability, you have to configure your rsync as a deamon. I used the following config file:

\begin{lstlisting}[caption=rsync Deaom config file]
use chroot = true
hosts allow = 0.0.0.0/0

transfer logging = true
log file = /var/log/rsyncd.log
log format = %h %o %f %l %b

[share]
comment = Public Share
path = /tmp/rsync
read only = no
list = yes
uid = nobody
gid = nobody
\end{lstlisting}

Then you are able to start the daemon via the following command:

\begin{lstlisting}[caption=Command to start rsync Daemon]
sudo rsync --config=/path/to/conf/rsyncd.conf --daemon --no-detach
\end{lstlisting}

For more information on the rsync daemon please refer to: \url{http://www.jveweb.net/en/archives/2011/01/running-rsync-as-a-daemon.html} and \url{http://wiki.ubuntuusers.de/rsync} (German).

After this configuration you can easily copy files to the host over the network (without a password prompt) via this command:

\begin{lstlisting}[caption=Copy files to the configured rsync share]
rsync -r -v /desired/path <ip>::<name of share>/desired/path
\end{lstlisting}

The \textit{-r} parameter means \textit{recursive} and the \textit{-v} means \textit{verbose}.